\documentclass[12pt,a4paper]{article}

% Packages for Vietnamese
\usepackage[utf8]{inputenc}
\usepackage[T1]{fontenc}
\usepackage[vietnamese]{babel}

% Essential packages
\usepackage{amsmath,amssymb,amsfonts}
\usepackage{graphicx}
\usepackage{cite}
\usepackage{url}
\usepackage{hyperref}
\usepackage{geometry}
\usepackage{fancyhdr}
\usepackage{setspace}

% Page setup
\geometry{margin=2.5cm}
\onehalfspacing
\pagestyle{fancy}
\fancyhf{}
\rhead{\thepage}
\lhead{\leftmark}

% Title and author information
\title{[Tiêu đề nghiên cứu]}
\author{[Tên tác giả]\\
\small [Tên trường/tổ chức]\\
\small \texttt{[email@domain.com]}}
\date{\today}

\begin{document}

\maketitle

\begin{abstract}
[Tóm tắt nghiên cứu 200-300 từ. Bao gồm mục tiêu, phương pháp, kết quả chính và kết luận.]
\end{abstract}

\textbf{Từ khóa:} [từ khóa 1, từ khóa 2, từ khóa 3]

\newpage
\tableofcontents
\newpage

\section{Giới thiệu}

\subsection{Bối cảnh nghiên cứu}
[Mô tả vấn đề nghiên cứu và tầm quan trọng]

\subsection{Tình hình nghiên cứu}
[Tổng quan các nghiên cứu liên quan]

\subsection{Khoảng trống nghiên cứu}
[Xác định vấn đề chưa được giải quyết]

\subsection{Mục tiêu và câu hỏi nghiên cứu}
[Mục tiêu cụ thể và câu hỏi nghiên cứu]

\section{Tổng quan tài liệu}

\subsection{Cơ sở lý thuyết}
[Các lý thuyết nền tảng]

\subsection{Các nghiên cứu liên quan}
[Phân tích các nghiên cứu trước đây]

\subsection{Khung lý thuyết}
[Khung lý thuyết cho nghiên cứu]

\section{Phương pháp nghiên cứu}

\subsection{Thiết kế nghiên cứu}
[Mô tả thiết kế nghiên cứu]

\subsection{Mẫu nghiên cứu}
[Mô tả đối tượng và mẫu nghiên cứu]

\subsection{Thu thập dữ liệu}
[Phương pháp thu thập dữ liệu]

\subsection{Phân tích dữ liệu}
[Phương pháp phân tích dữ liệu]

\section{Kết quả}

\subsection{Thống kê mô tả}
[Thống kê cơ bản về dữ liệu]

% Ví dụ về bảng
\begin{table}[h]
\centering
\caption{Mô tả bảng}
\label{tab:example}
\begin{tabular}{|l|c|c|}
\hline
Biến số & Giá trị & Ghi chú \\
\hline
Biến 1 & 123 & Đơn vị \\
Biến 2 & 456 & Đơn vị \\
\hline
\end{tabular}
\end{table}

\subsection{Kết quả phân tích chính}
[Kết quả chính của nghiên cứu]

% Ví dụ về hình ảnh
\begin{figure}[h]
\centering
% \includegraphics[width=0.8\textwidth]{figure1.png}
\caption{Mô tả hình ảnh}
\label{fig:example}
\end{figure}

\subsection{Kết quả phân tích bổ sung}
[Các phân tích bổ sung]

\section{Thảo luận}

\subsection{Giải thích kết quả}
[Giải thích ý nghĩa của kết quả]

\subsection{So sánh với nghiên cứu trước}
[So sánh với các nghiên cứu khác, ví dụ: theo Smith et al. \cite{smith2023}]

\subsection{Hạn chế của nghiên cứu}
[Những hạn chế và điểm yếu]

\subsection{Gợi ý nghiên cứu tương lai}
[Hướng nghiên cứu tiếp theo]

\section{Kết luận}

\subsection{Tóm tắt kết quả chính}
[Tóm tắt các phát hiện quan trọng]

\subsection{Đóng góp của nghiên cứu}
[Đóng góp về mặt lý thuyết và thực tiễn]

\subsection{Khuyến nghị}
[Các khuyến nghị dựa trên kết quả]

% Bibliography
\bibliographystyle{plain}
\bibliography{references}

% Nếu không có file .bib, có thể sử dụng thebibliography
% \begin{thebibliography}{99}
% \bibitem{smith2023}
% Smith, J., Doe, A. (2023). Title of the paper. \textit{Journal Name}, 15(2), 123-145.
% \end{thebibliography}

\appendix
\section{Phụ lục A}
[Nội dung phụ lục]

\end{document}

% Hướng dẫn biên dịch:
% 1. Cài đặt LaTeX (TeX Live, MiKTeX, hoặc MacTeX)
% 2. Chạy: pdflatex template-research-paper.tex
% 3. Nếu có tài liệu tham khảo: bibtex template-research-paper
% 4. Chạy lại pdflatex hai lần để cập nhật tham chiếu

% Ghi chú về việc sử dụng AI trong Cursor IDE:
% - Có thể yêu cầu AI viết các phần cụ thể bằng LaTeX
% - Prompt ví dụ: "Hãy viết phần giới thiệu bằng LaTeX cho nghiên cứu về [chủ đề]"
% - AI có thể giúp format công thức toán học, bảng, và hình ảnh
% - Luôn kiểm tra syntax LaTeX sau khi AI tạo ra
